% RESUMO--------------------------------------------------------------------------------

\begin{resumo}[RESUMO]
\begin{SingleSpacing}

% Não altere esta seção do texto--------------------------------------------------------
\imprimirautorcitacao. \imprimirtitulo. \imprimirdata. \pageref {LastPage} f. \imprimirprojeto\ – \imprimirprograma, \imprimirinstituicao. \imprimirlocal, \imprimirdata.\\
%---------------------------------------------------------------------------------------

O presente trabalho refere-se ao desenvolvimento do protótipo de uma matriz de LED que será parte da exposição interativa ``Luz, Ciência e Emoção'', idealizada pela arquiteta Dra. Maristela Mitsuko Ono e pelo engenheiro Dr. James Alexandre Baraniuk. A matriz, chamada de ``Mesa de Bolinhas'', se enquadra no setor artístico da exposição, que traz experimentos envolvendo os conceitos de luz trabalhados nos ensinos pré-escolar e fundamental, e proporcionará uma experiência tangível-visual impactante aos observadores, causando deslumbramento e entusiamos através da interação com arte e tecnologia.

O projeto conta com técnicas de \emph{software} e \emph{hardware} para endereçar 129 LEDs RGBs e controlar seus 387 canais de PWM, mapear 128 sensores reflexivos e submetê-los a rotinas de \emph{debouncing}, implementar máquina de estados finita em \emph{firmware}, desenvolver um teclado analógico com apenas uma GPIO, projetar um conversor de nível lógico não-inversor com um único transistor, entre outras funcionalidades.\\

\textbf{Palavras-chave}: Matriz de LED. Matriz de sensores. Arte generativa.

\end{SingleSpacing}
\end{resumo}

% OBSERVAÇÕES---------------------------------------------------------------------------
% Altere o texto inserindo o Resumo do seu trabalho.
% Escolha de 3 a 5 palavras ou termos que descrevam bem o seu trabalho 

